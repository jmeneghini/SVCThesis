\chapter{Monte Carlo Approach}

\section{Transport Theory} \label{sec:theory}
\par To represent the computational domain discretely, space is broken up into a grid of voxels (volume pixel), with each voxel being assigned a particular material depending on the geometries and compounds/elements in the domain. Within our discrete space, given a photon position $\va*{r}$, the corresponding voxel in which the photon resides can be calculated. Therefore, all possible $\va*{r}$'s can be assigned a particular material $M$ in the domain.

\par A photon's position after taking the $n$-th step in the domain, $\va*{r}_n$, is represented by the parametric ray equation:
\begin{equation} \label{eq:1}
    \va*{r}_n = \va*{r}_{n-1} + \vu*{d}t_n,
\end{equation}
where $\va*{r}_{n-1}$ is the initial position before the $n$-th step, $\vu*{d}$ is a unit vector in the direction of the step, and $t_n$ is the length of the $n$-th step.

\par To randomly sample $t_n$ in a homogeneous domain, we utilize the following probability density function (PDF) $p(t)$ of the distance traveled $t$ by a photon of energy $E$ through material $M$ before interacting:
\begin{equation} \label{eq:2}
    p(t) = n\sigma \exp\left[-t(n\sigma)\right],
\end{equation}
where $n$ is the number density of $M$ and $\sigma = \sigma(E, M)$ is the microscopic cross-section of $M$ at $E$.
\par Using the inversion method for sampling a PDF on Equation \ref{eq:2}, random values of the free path $t$ can be generated by
\begin{equation}
    t = -\frac{1}{n\sigma} \ln \gamma, 
\end{equation}
    %inlining for space concerns
where $\gamma$ is a uniformly distributed random number in the interval $[0, 1)$. This value of $t$ is sampled for each step and is used as $t_n$ in Equation \ref{eq:1} to determine the length of the $n$-th step \cite{vassiliev_monte_2017}.


\subsection{Surface \& $\delta$-Tracking}\label{ssec:delta_tracking}

\par If, after taking a step, the photon lands in a voxel with a different material (an inhomogeneous domain), then the corresponding free path for the new material must be accounted for. One method, called surface-tracking, requires photons to be stopped at voxel boundaries and further necessitates intersections with surrounding voxels to be calculated, which can be computationally intensive for materials that have a large average free path. 
\par Alternatively, the $\delta$-tracking algorithm offers a solution by sampling the maximum cross-section $\sigma_{\text{max}}$ in the computational domain. This, in turn, brings down the average free path to the minimum in the domain. To account for this decrease in free path, the algorithm introduces $\delta$ interactions as an alternative to real interactions, resulting in no change to the energy or direction. The probability $P_{\delta}$ of a $\delta$ interaction is given by 
\begin{equation}
    P_{\delta} = \frac{\sigma_{\text{max}}(E) - \sigma(E, M)}{\sigma_{\text{max}}(E)},
\end{equation}

where $E$ is the energy of the photon undergoing the step. When the photon lands in the material corresponding to the maximum cross-section, $\sigma(E, M) = \sigma_{\text{max}}$ and $P_{\delta} = 0$. On the contrary, if the photon landed in air and the domain's maximum cross-section corresponded to lead, then $\sigma(E, M) << \sigma_{\text{max}}$, making $P_{\sigma} \approx 1$.

\par Overall, $\delta$-tracking is significantly more computationally efficient for domains with similar cross-sections and can be shown to yield equivalent results to surface-tracking \cite{vassiliev_monte_2017}.

\subsection{Photon-Matter Interactions}
\par If a $\delta$ interaction does not occur, then a real interaction is sampled. Therefore, the probability of a real interaction $P_r$ is directly related to $P_\delta$ by
\begin{equation}
    P_r = 1 - P_\delta.
\end{equation}
\par If a real interaction occurs in material $M$, then the probability of interaction $i$ occurring is 
\begin{equation}
P_i = \frac{\sigma_i(E, M)}{\sigma (E, M)}
\end{equation}
where $\sigma_i$ is the cross-section of interaction $i$. If there are $N$ possible interactions for a particular $E$ and $M$, then $\sigma (E, M)$ is simply the sum of each interaction's cross-section.

\par For x-rays, there are three possible photon interactions: photoelectric effect, coherent scattering, and incoherent scattering.

\subsubsection{Photoelectric Effect}

\par In the photoelectric effect model used in MIDSX, a rather simple approach is taken. When a photon interacts with an atom's electron, the photon is terminated and all energy is deposited at the location of interaction. In general purpose particle transport code systems, when a photoelectric interaction occurs, a photon of energy $E$ is absorbed by an electron in subshell $i$, causing the electron to leave the atom with energy $E_e = E - U_i$, where $U_i$ is the binding energy of the $i$th subshell. In addition, photons are emitted due to atomic relaxations. For photon energies in the medical imaging range (30 - 120 keV), the energy of the released electrons does not allow for significant traversal through typically encountered materials, such as tissue, bone, and fat. This limited traversal results in a localized dose distribution, in turn, validating the model used by MIDSX.

\subsubsection{Coherent Scattering}

\par Thomson scattering is defined as an incoming photon of energy $E$ elastically scattering with a free electron at rest, resulting in a scattered photon of same energy $E$. The atomic differential cross-section (DCS) per unit solid angle $\Omega$ for the interaction can be derived with classical electrodynamics, and is given by

\begin{equation}
    \frac{d\sigma_T}{d\Omega} = r_e^2 \frac{1 + \cos^2(\theta)}{2},
\end{equation}
where $r_e^2$ is the classical electron radius.
\par In an atom, photons scatter off bound electrons rather than the free electrons described by Thomson scattering, resulting in what is known as coherent (Rayleigh) scattering. The DCS per unit solid angle $\Omega$ of the interaction, ignoring absorption edge effects, is given by

\begin{equation}
    \frac{d\sigma_{Co}}{d\Omega} = \frac{d\sigma_T}{d\Omega} F(x, Z),
    \label{eq:DCS_omega_rayleigh}
\end{equation}

where $x$ is the momentum transfer between the photon and atom, $Z$ is the atomic number of the atom, and $F(x, Z)$ is the atomic form factor. $x$ is related to the scattering angle $\theta$ by

\begin{equation}
    x = ak\sqrt{1 - \cos\theta},
\end{equation}

where

\begin{align}
    a = \frac{m_e c^2}{\sqrt{2}hc} && \textrm{and} && k = \frac{E}{m_e c^2},
\end{align}

\par In these definitions, $m_e$ is the mass of an electron, $c$ is the speed of light, and $h$ is Planck's constant.

\par Equation~\ref{eq:DCS_omega_rayleigh} can be integrated over $\phi$ to obtain the DCS per unit polar angle $\theta$:

\begin{equation}
    \frac{d\sigma_{Co}}{d\theta} = \pi r_e^2 \sin \theta (1 + \cos^2 \theta) F(x, Z)^2.
\end{equation}

\par The PDF of the polar angle $\theta$ is then given by

\begin{equation}
    p(\theta) d\theta = \frac{d\sigma_{Co}}{d\theta} \frac{1}{\sigma_{Co}} d\theta = \frac{\pi r_e^2}{\sigma_{Co}} \sin \theta (1 + \cos^2 \theta) F(x, Z)^2 d \theta.
\end{equation}

\par The PDF of $\theta$ can be transformed into a PDF of $\mu = \cos \theta$, resulting in

\begin{equation}
    p(\mu) = \frac{\pi r_e^2}{\sigma_{Co}} (1 + \mu^2) F(x, Z)^2.
\end{equation}

\par To then sample $\mu$ for a particular scattering event, the inversion method is used. In particular, a look up table for the CDF of $P(\mu)$ is generated for each material in the domain for a grid of $\mu$ values. The details of this algorithm are discussed in Appendix ...

% \begin{large}
%     \bf{3. Incoherent Scattering}
% \end{large}
\subsubsection{Incoherent Scattering}

\par Incoherent (Compton) scattering is defined as an incoming photon of energy $E$ interacting with an atom's electron, resulting in a scattered photon of energy $E'$ and a released electron with energy $E_e = E - E' - U_i$, where $U_i$ is the binding energy of the interacting subshell. While coherent scattering effectively interacts with the entire electron cloud, incoherent scattering interacts with an individual electron. The DCS per unit solid angle $\Omega$ of the interaction was derived by Klein and Nishina in 1929, making it one of the first results of quantum electrodynamics. The Klein-Nishina formula is given by

\begin{equation}
    \frac{d\sigma_{KN}}{d\Omega} = \frac{r_e^2}{2} \left(\frac{E'}{E}\right)^2 \left(\frac{E'}{E} + \frac{E}{E'} - \sin^2 \theta \right),
    \label{eq:DCS_omega_KN}
\end{equation}

\par Note that when $E' = E$, the Equation~\ref{eq:DCS_omega_KN} simplifies to Equation~\ref{eq:DCS_omega_rayleigh}, showing that incoherent scattering is a generalization of coherent scattering for inelastic interactions.

\par Applying conservation of energy and momentum to a free electron at rest, the following equation can be derived relating the scattered photon energy $E'$ to the scattering angle $\theta$ and the incident photon energy $E$:

\begin{equation}
    E' = \frac{E}{1 + k(1 - \cos \theta)}.
\end{equation}

\par Similar to the Thomson DCS, the Klein-Nishina DCS assumes a free electron at rest. In an atom, the electron is bound, resulting in a modified DCS. In the case of incoherent scattering, the Klein-Nishina DCS is modified by the Coherent Scattering Function $S(x, Z)$, making the DCS per unit solid angle $\Omega$ of the interaction

\begin{equation}
    \frac{d\sigma_{In}}{d\Omega} = \frac{d\sigma_{KN}}{d\Omega} S(x, Z).
\end{equation}

\par Instead of directly sampling the PDF of the DCS, $\mu$ is first sampled from Equation~\ref{eq:DCS_omega_KN} using the acceptance-rejection method developed by Kahn, then $S(x, Z)$ is sampled with the standard acceptance-rejection method. The details of this algorithm are discussed in Appendix ...

\section{Methods}

In order to simulate the transport of x-rays through a domain, the following is required:

\begin{enumerate}
    \item The geometry of the domain
    \item The materials in the domain
    \item The associated cross-sections, form factors, and scattering functions of the materials
    \item The source of the x-rays (energy spectrum, position, and direction)
    \item The number of photons to simulate
\end{enumerate}

and to retrieve information about the performed simulation, the following is required:

\begin{enumerate}
    \item Geometries to check for intersection
    \item Quantities to tally
    \item Derived quantities to calculate from tallied quantities
\end{enumerate}

\subsection{Geometry}
\par In MIDSX, the computational domain is composed of 3D arrays of voxels, with each voxel being assigned a material ID. The domain is assigned a particular size, indicated by its spatial extent along the x, y, and z dimensions. In addition, the domain is assigned a background material ID, which assigns a material to all space in the domain not occupied by voxels. The voxelized geometries inside the domain are specified by NIFTI files \cite{nifti2004}. These NIFTI files are assigned material IDs, spatial size, voxel size, and an origin, which is the location of the geometry in the domain that corresponds to the origin of the NIFTI file. After constructing the NIFTI files, the domain is defined by specifying the background material ID, domain size, and a list of NIFTI files in a .JSON file, which is then read by the MIDSX executable.
\par In the code, both a \mintinline{c++}{VoxelGrid} and \mintinline{c++}{ComputationalDomain} object are created, with the \mintinline{c++}{ComputationalDomain} consisting of the specified dimensions, background material ID, and a vector of \mintinline{c++}{VoxelGrid} objects which are created via the provided NIFTI files. To determine the current material of a photon, the code first checks if the photon is inside the \mintinline{c++}{ComputationalDomain}. Should this be true, it then checks if the photon is inside any of the \mintinline{c++}{VoxelGrid} objects. If so, it is then determined which \mintinline{c++}{Voxel} object within the \mintinline{c++}{VoxelGrid} the photon is located inside, then the material ID of the corresponding voxel is returned. If the photon is not inside the \mintinline{c++}{ComputationalDomain}, then the photon is terminated.

\subsection{Materials \& Data}
\par In MIDSX, materials are all defined in an SQLite database \cite{sqlite2020hipp}, which is interfaced by the \mintinline{c++}{DataAccessObject}. The database contains the following information for each element:

\begin{enumerate}
    \item Symbol
    \item Atomic Number
    \item Mass
    \item Mass Density
    \item Number Density
    \item Mass Number
\end{enumerate}

\par All of the above data was obtained from the periodictable \cite{periodictable2022} and mendeleev \cite{mendeleev2021} Python packages. In addition, the database contains the following data from the Electron Photon Interaction Cross Sections (EPICS2017) library \cite{cullen_survey_nodate} for each element:

\begin{enumerate}
    \item Total Microscopic Cross-Section
    \item Photoelectric Microscopic Cross-Section
    \item Coherent Scattering Microscopic Cross-Section
    \item Incoherent Scattering Microscopic Cross-Section
    \item Atomic Form Factor
    \item Scattering Function
\end{enumerate}

\par In MIDSX, all the above data is initialized upon created creation of the \mintinline{c++}{InteractionData} object, which requires a vector of strings of materials names. For convience, the \mintinline{c++}{ComputationalDomain} implements a helper function to generate \mintinline{c++}{InteractionData} for the materials in the domain. The material names correspond to entries in the SQLite database, which contains a table of material compositions and mass densities which were obtained from NIST's mass attenuation coefficient database \cite{hubbell_x-ray_2004}. The \mintinline{c++}{InteractionData} object contains a map of \mintinline{c++}{Material} objects with their names, along with additional computed data, such as the maximum cross-section used in delta-tracking. The \mintinline{c++}{Material} objects construct the above data for the specified material by performing the additivity approximation described by the generalization of Equation~\ref{eq:mu_weighted} for any data of interest, rather than specifically $\mu_m$. The data is then separated into two further objects: \mintinline{c++}{MaterialData} and \mintinline{c++}{MaterialProperties}. \mintinline{c++}{MaterialData} contains the microscopic cross-sections, form factors, and scattering functions, while \mintinline{c++}{MaterialProperties} contains the mass density, number density, and mass.
\par In addition to retrieving and storing the above data, the \mintinline{c++}{MaterialData} object constructs interpolators for all its data. The interpolators for each type of data vary depending on its shape. For example, the photoelectric and total cross-sections are interpolated with a log-log linear interpolator, while the incoherent and coherent cross-sections are interpolated with a log-log cubic spline interpolator.

\subsection{Source}
\par To generate the initial position, direction, and energy of a photon, a \mintinline{c++}{PhotonSource} object is initialized with a \mintinline{c++}{SourceGeometry}, \mintinline{c++}{Directionality}, and an \mintinline{c++}{EnergySpectrum} object. The three initializing objects are virtual classes, allowing the user to specify the attributes of the source. The inheritance structure is shown in Fig (). 


\subsection{Tallies}
\par To measure simulation data, one must first decide when to trigger the measurement. In particular, MIDSX supports both surface and volume geometries that trigger when a photon passes through or enters the geometry, respectfully. For surfaces, users can choose discs and rectangles, while for volumes, there are cuboids. In the code, these geometries are defined by \mintinline{c++}{Tally} objects.
\par At the start of the simulations, users can specify quantities that they want measured upon the trigger of a \mintinline{c++}{Tally}. For surfaces, these include incident photon energy, entrance cosine, and number of photons. For volumes, in addition to their own implementation of incident energy and number of photons, energy deposition and number of interactions are available for measurement. Furthermore, each quantity has the ability to group together measurements that were triggered by photons that underwent a single coherent scatter, a single incoherent scatter, multiple scatters, and no scatters. While this does noticeably increase the computation time, it was necessary to validate the interaction models discussed in Section \ref{sec:theory}.
\par At the end of simulation, one might want to calculate additional derived quantities from the tallied quantities from the simulation. With this mind, MIDSX contains a \mintinline{c++}{DerivedQuantity} object that can calculate planar fluence and air kerma, which were implemented for use in half value layer simulations.
\vspace{1cm}
\par Using the described theory and methodology, the \mintinline{c++}{RunSimulation} function transports $N$ photons by calling \mintinline{c++}{PhysicsEngine} $N$ times. This parallel execution is facilitated through the use of the OpenMP API \cite{dagum1998openmp}, which distributes the computational load across multiple processors to increase performance. The steps taken by \mintinline{c++}{PhysicsEngine} to transport a single photon is summarized in Figure~\ref{fig:PhysicsEngineFlowChart}.


\begin{figure}[H]
    \centering
	\includegraphics[width=0.9\textwidth]{chapter3/physics_engine_flow_chart.pdf}
	\caption{Photon transport process in MIDSX.}
	\label{fig:PhysicsEngineFlowChart}
\end{figure}

\section{Validation}
\par To validate the accuracy of MIDSX, validation simulations were performed and compared to reference data (PENELOPE, EGSrnc, Geant4, and MCNP) obtained by the American Association of Physicists in Medicine Task Group Report 195 (TG-195) \cite{sechopoulos_monte_2015}. The simulations performed from TG-195 were Case 1: "Half Value Layer," Case 2: "Radiography and Body Tomosynthesis," and Case 5: "CT with a Voxelized Solid." 

\par For Case 1, depicted in Figure~\ref{fig:case1}, the primary air kerma was measured on a far away, circular region of interest (ROI) with a cone beam point source collimated such that all primary particles would be incident upon the ROI. The primary air kerma was measured with the domain filled only with air and then compared to the measured air kerma with an aluminum filter of thickness $t$ placed between the source and ROI. The ratios of the half value layer (HVL) and quarter value layer (QVL) primary air kerma to the primary background air kerma is represented by $R_1$ and $R_2$, respectively. By setting $t$ to correspond to the HVL and QVL for a particular spectrum, one can validate the material attenuation properties of an MC code system by comparing the simulated $R_1$ and $R_2$ to their theoretical values of 0.5 and 0.25, respectively. The simulation was performed for the monoenergetic energies of 30 keV and 100 keV, along with the polyenergetic spectrums of 30 kVp and 100 kVp, which were provided by TG-195. MIDSX's results for Case 1 agree to within 0.39\% of the mean results published by TG-195 (not shown).

\begin{figure}[H]
    \centering
    \begin{subfigure}[l]{\textwidth}
        \includegraphics[width=\textwidth]{chapter3/case1_top_view.pdf}
        \caption{Top View}
    \end{subfigure}
    \hfill % optional spacing
    \begin{subfigure}[l]{\textwidth}
        \includegraphics[width=\textwidth]{chapter3/case1_perspective_view.png}
        \caption{Perspective View}
    \end{subfigure}
    \caption{A (a) top and (b) perspective view of the geometry described by Case 1.}
    \label{fig:case1}
\end{figure}

\par For Case 2, a full-field and pencil beam x-ray source were directed towards a cuboid tissue phantom at $0^\circ$ and $15^\circ$ from an axis drawn perpendicularly from the center of the cuboid. Directly behind and inside the phantom, a grid of square ROIs and cube volumes of interests (VOI) were placed, respectively. The simulation was performed for the TG-195-provided polyenergetic spectrum of 120 kVp and its mean energy of 56.4 keV. For the $0^\circ$, full-field ROI measurements (not shown), a $<3.9$\% mean percent error (MPE) is seen for MIDSX's results to each ROI simulation. Furthermore, for the $0^\circ$ pencil-beam ROI measurements shown in Figure~\ref{fig:ROIPGraph}, a $<2.1$\% MPE is observed for each ROI simulation except for the case of a single incoherent scatter. In this particular case, MIDSX's results for ROI 4 and 5 are significantly lower, with the MPE reaching 13.1\% for ROI 5. The full-field VOI energy deposition measurements depicted in Figure~\ref{fig:BDGraph} show a minimal MPE of less than 0.1\% for the $0^\circ$ source. Conversely, for the MIDSX results at $15^\circ$, the MPE reaches an unexpectedly larger value of approximately 0.6\%.

\begin{figure}
    \centering
    \includegraphics[width=\textwidth]{chapter3/case2_perspective_view.png}
    \caption{A perspective view of the geometry described by Case 2.}
\end{figure}

\par For Case 5, a fan beam was collimated to the center of a voxelized human torso phantom provided by TG-195. To replicate a CT image, the simulation was repeated for several angles along a circle surrounding the phantom. The simulation was performed with Case 2's 120 kVp energy spectrum, and energy deposition was measured in the different materials/organs composing the phantom. Almost all of MIDSX's results for the $0^\circ$ source presented in Figure~\ref{fig:CTGraph} are systematically lower than the mean of the reference code systems, with MPE's ranging from 1.1\% to 6.3\%. This pattern is disrupted by the thyroid, which is larger than the mean by 2.9\%. 


\begin{figure}[H]
    \centering
	\includegraphics[width=1.0\textwidth]{chapter3/HVL_and_QVL_paper_ready.pdf}
	\caption{Results for the (a) HVL and (b) QVL simulations as described by Case 1. The ratios of the primary HVL and QVL air kermas to the primary background air kermas is represented by $R_1$ and $R_2$, respectively. The simulation was performed for the monoenergetic energies 30 keV and 100 keV, along with the polyenergetic spectrums of 30 kVp and 100 kVp provided by TG-195. A dashed-line is placed at $R_1 = 0.5$ and $R_2 = 0.25$ for comparison.}
	\label{fig:HVLGraph}
\end{figure}

\begin{figure}[H]
    \centering
	\includegraphics[width=1.0\textwidth]{chapter3/radiography_body_dep_paper_ready.pdf}
	\caption{The energy deposited per initial photon (p.i.p.) (eV/photon) in the simulated tissue for the full-field simulation as described by Case 2. The simulation was performed at 56.4 keV and 120 kVp at both (a) $0^\circ$ and (b) $15^\circ$, with the 120 kVp spectrum provided by TG-195.}
 	\label{fig:BDGraph}
\end{figure}


\begin{figure}[H]
    \centering
	\includegraphics[width=1.0\textwidth]{chapter3/ROI_0_deg_paper_ready.pdf}
	\caption{The energy per initial photon (eV/photon) of photons incident upon each region of interest (ROI) for the $0^\circ$, full-field, 56.4 keV simulation as described by Case 2. The incident energy was determined separately for photons that underwent (a) no real interactions, (b) a single incoherent scatter, (c) a single coherent scatter, (d) and multiple scatters.}
	\label{fig:ROIFFGraph}
\end{figure}

\begin{figure}[H]
    \centering
	\includegraphics[width=1.0\textwidth]{chapter3/ROI_0_deg__pencil_paper_ready_wo_mutiple.pdf}
	\caption{The energy per initial photon (p.i.p.) (eV/photon) of photons incident upon each region of interest (ROI) for the $0^\circ$, pencil beam, 56.4 keV simulation as described by Case 2. The incident energy was determined separately for photons that underwent (a) a single incoherent scatter and (b) a single coherent scatter.}
	\label{fig:ROIPGraph}
\end{figure}

\begin{figure}[H]
    \centering
	\includegraphics[width=1.0\textwidth]{chapter3/VOI_paper_ready.pdf}
	\caption{The energy deposited per initial photon (eV/photon) in the volumes of interests (VOIs) for the full field simulation as described by Case 2. The simulation was performed at (a) $0^\circ$/56.4 keV, (b) $0^\circ$/120 kVp, (c) $15^\circ$/56.4 keV, and (d) $15^\circ$/120 kVp.}
	\label{fig:VOIFFGraph}
\end{figure}

\begin{figure}[H]
    \centering
	\includegraphics[width=1.0\textwidth]{chapter3/CT_564_180.pdf}
	\caption{The energy deposited per initial photon (eV/photon) in the material IDs for the $180^\circ$, 56.4 keV simulation as described by Case 5.}
	\label{fig:CTGraph}
\end{figure}

\begin{figure}[H]
    \centering
	\includegraphics[width=1.0\textwidth]{chapter3/CT_120_0_big_font.pdf}
	\caption{The energy deposited per initial photon (p.i.p.) (eV/photon) in the material IDs/organs composing a voxelized human phantom for the $0^\circ$, 120 kVp simulation as described by Case 5.}
	\label{fig:CTGraph}
\end{figure}

\section{Discussion}
\par Overall, MIDSX shows varied but reasonable agreement with the reference code systems of TG-195. For the HVL layer measumrents described by Case 1, shown in Figure~\ref{fig:HVLGraph}, statistical agreement is seen with all code systems except for PENELOPE in the HVL and QVL 100 keV simulation. In addition, for the 30 keV simulation, statistical agreement is seen with PENELOPE, Geant4, and MCNP for the HVL, along with PENELOPE and MCNP for the QVL. Note that while no statistical agreement is observed for the other energies/thicknesses, all MIDSX results have a mean percent error (MPE) of $<0.32$\%.

\par For the ROIs of Case 2, agreement is seen almost universally, except for the single incoherent scatter deposition energy in ROI 4 and 5 for the pencil beam source shown in Figure~\ref{fig:ROIPGraph}. In particular, the results reached a max MPE of 13.1\% for ROI 5, indicating a potential error in the incoherent scattering energy/angular distribution sampling algorithm. This discrepancy is likely a result of an error in the rejection sampling algorithm employed by MIDSX. While this algorithm shows agreement for the full-field ROI 5, the geometry of the ROI, combined with the pencil beam, results in only narrow-angle scatters hitting the ROI. Since the scattering angle distribution of incoherent scattering at the medical imaging energy range becomes extremely steep at the scattering angle $\theta = 0^\circ$, there is likely numerical instability presenting itself in the algorithm that needs to be analyzed. In addition, the $15^\circ$ full-field tissue energy deposition measurements, shown in Figure~\ref{fig:BDGraph}, were larger than the reference code systems', with an MPE of 0.6\%. With the MPE increasing significantly from the $0^\circ$ to $15^\circ$ measurements, this hints at a possible geometric error with the source and/or body. However, the source's position and angular distribution, along with the domain's and tissue's dimensions, were carefully verified, making the scene geometry unlikely to be the source of discrepancy.

\par For Case 5, almost all organ energy deposition results were lower than the reference code results, with the MPE reaching 6.3\%, except for the thyroid, with an MPE of 2.9\% larger. One common error reported by TG-195 that could result in the observed discrepancies is the incorrect orientation of the voxelized phantom in the computational domain. The orientation was verified by taking the root mean square percent error (RMSPE) of the MIDSX data with respect to the results of each simulated angle reported by TG-195. As expected, the RMSPE with respect to $0^\circ$ was the minimum, verifying that the phantom's orientation during the CT simulation was correct.

\par Despite verifying the orientation, the deviation of MIDSX's energy deposition results for both Cases 2 and 5 suggest that there may be other underlying errors in the MIDSX system that need further investigation. Potential factors could include the software's handling of scattering events, cross-section data initialization, and interpolation. Future work will study these aspects to pinpoint and rectify the source of the systematic errors observed in the MIDSX results.
% \par Overall, MIDSX shows relative agreement with the PENELOPE, EGSnrc, Geant4, and MCNP results provided by TG-195 for the three examined cases. For the HVL layer simulation described by Case 1, statistical agreement is seen with all code systems except for PENELOPE for the HVL and QVL 100 keV simulation. In addition, for the 30 keV simulation, statistical agreement is seen with PENELOPE, Geant4, and MCNP for HVL, along with PENELOPE and MCNP for QVL. Note that while no statistical agreement is observed for the other energies/thicknesses, all MIDSX results are within 0.32\% of the mean of the reference code systems.

% \par For Case 2, agreement is rather varied. For the full-field ROI measurements not shown in this paper due page constraints, very little statistical agreement is seen between the code systems; however, a $<3$\% mean percent error (MPE) is seen for MIDSX's results to each ROI simulation. Furthermore, for the pencil-beam ROI measurements shown in Fig \ref{fig:ROIPGraph}, statistical agreement is not readily observed, but a $<2.1$\% MPE is observed for each ROI simulation expect for the case of a single incoherent scatter. In this particular case, MIDSX's results for ROI 4 and 5 are significantly lower, with the MPE reaching 10\% for ROI 5. This discrepancy is likely a result of an error in the rejection sampling algorithm employed by MIDSX. While this algorithm shows agreement for the full-field ROI 5, the geometry of the ROI, combined with the pencil beam, results in only narrow angle scatters hitting the ROI. Since the scattering angle distribution of incoherent scattering at the medical imaging energy range contains a vertical asymptote approaching 0 at $\theta = 0^\circ$, there is likely some form of numerical instability presenting itself in the algorithm that needs to be analyzed.

% \par In the full-field tissue deposition measurements depicted in Fig \ref{fig:BDGraph}, we do not observe statistical agreement. However, for the $0^\circ$ case, the disagreement between code systems is minimal with an MPE of less than 0.1\% for MIDSX. Conversely, for the MIDSX results at $15^\circ$, the MPE reaches approximately 0.5\%. Despite extensive investigations into this pronounced discrepancy, a solution remains elusive.

% \par For Case 5, almost all of MIDSX's results are marginally lower than the mean of the reference code systems, with MPE's ranging from 1.1\% to 6.3\%. This pattern is disrupted by the thyroid, which is larger than the mean by 2.9\%. In order to quantify the cumulative error, the root mean square percent error (RMSPE) was calculated using each organ result, which resulted in the RMSPE for MIDSX being 5\%. In addition, with all other code systems typically having an MPE less than 1\%, except for MCNP which reaches an MPE of 2.2\% for the breast, there appears to a systematic error with the MIDSX code system with regard to the CT simulation. One common error reported by TG-195 is the incorrect orientation of the voxelized phantom in the computational domain. On top of checking the scenes geometry and the .comp file, the orientation was verified by taking the RMSPE of the MIDSX data with respect to the results of each simulated angle reported by TG-195. As expected, the RMSPE with respect the $0^\circ$ was the minimum, further solidifying the belief that the phantom's orientation during the CT simulation is accurate.

% \par However, despite the orientation being verified, the consistent deviation of MIDSX energy deposition results for both Case 2 and 5 raises concerns. This suggests that there may be other underlying issues or intricacies in the MIDSX system that need further investigation. Potential factors could include the software's handling of certain physics processes, voxel resolution, or computational approximations. It's imperative for future research to delve deeper into these aspects to pinpoint and rectify the source of the systematic errors observed in the MIDSX results.


